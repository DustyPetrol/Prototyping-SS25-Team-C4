\documentclass[conference]{IEEEtran}
\IEEEoverridecommandlockouts

\usepackage{cite}
\usepackage{amsmath,amssymb,amsfonts}
\usepackage{algorithmic}
\usepackage{graphicx}
\usepackage{textcomp}
\usepackage{xcolor}
\usepackage{hyperref}
\usepackage{listings}

\def\BibTeX{{\rm B\kern-.05em{\sc i\kern-.025em b}\kern-.08em
    T\kern-.1667em\lower.7ex\hbox{E}\kern-.125emX}}

\begin{document}

\title{Design and Implementation of a Line Following and Obstacle Avoidance Robot}

\author{
\IEEEauthorblockN{Your Name\IEEEauthorrefmark{1}, Student 2\IEEEauthorrefmark{1}, Student 3\IEEEauthorrefmark{1}, Student 4\IEEEauthorrefmark{1}}
\IEEEauthorblockA{\IEEEauthorrefmark{1}Systems Engineering and Prototyping, BSc Electronic Engineering\\
Hochschule Hamm-Lippstadt, Lippstadt, Germany\\
\{your.name, student2, student3, student4\}@stud.hshl.de}
}

\maketitle

\begin{abstract}
This paper presents the.
\end{abstract}

\begin{IEEEkeywords}
Line follower robot,
\end{IEEEkeywords}

\section{Introduction}

\section{System Design and Engineering}

\section{Prototyping and Design}

\subsection{Electronic Components}

\subsection{Virtual and Physical Prototypes}


\subsection{Hardware Assembly}
All components were mounted on a chassis and connected as per the schematic.

\section{Circuit Design}
%\begin{itemize}
 %   \item The schematic includes sensors, motor driver, Arduino, and power source.
  %  \item Wiring includes proper voltage connections and pin mappings for sensors and motors.
%\end{itemize}

\begin{figure}[htbp]
\centerline{\includegraphics[width=0.9\linewidth]{circuit_diagram.png}}
\caption{Complete circuit layout of the robot}
\label{fig:circuit}
\end{figure}

\section{Software Implementation}
\label{soft}

\subsection{Initialization of Sensors and Actuators}
Pins for sensors were defined and initialized in the `setup()` function. Below is a snippet:
\begin{lstlisting}[language=C++, caption=Sensor Initialization Code]
void setup() {
  pinMode(2, INPUT);  // IR sensor
  pinMode(9, OUTPUT); // Motor
  Serial.begin(9600);
}
\end{lstlisting}

\subsection{Movement Control Algorithms}


\subsection{Behavioral Decision Logic}


\section{Git Usage and Collaboration}
Project development was tracked using GitHub. Each member's contribution should be explained.

\begin{table}[ht]
\centering
\begin{tabular}{l|c}
\hline
\textbf{Team Member} & \textbf{Number of Commits} \\
\hline
Your Name     & 77 \\
Student 2     & 25 \\
Student 3     & 9  \\
Student 4     & 3  \\
\hline
\textbf{Total} & \textbf{114} \\
\hline
\end{tabular}
\caption{GitHub Commits by Team Members}
\label{table:commit}
\end{table}

\section{Key Achievements}


\section*{Acknowledgment}

\begin{thebibliography}{00}
\bibitem{arduino} Arduino Documentation. \url{https://www.arduino.cc/en/Guide}
\bibitem{tinkercad} Tinkercad Simulation. \url{https://www.tinkercad.com}
\bibitem{uppaal} Uppaal Model Checker. \url{http://www.uppaal.org/}
\end{thebibliography}

\section*{Declaration of Originality}
We hereby declare that this report is our own work and that we have acknowledged all sources used.

\end{document}
